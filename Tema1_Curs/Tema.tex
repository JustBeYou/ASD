\documentclass[12pt]{article}
\usepackage{design_ASC}
\usepackage{amsmath}
\usepackage{mathtools}
\usepackage{listings}

\setlength\parindent{0pt} %% Do not touch this

\title{Tema 1}

\author{Mihail Feraru\\
Grupa 142 - Structuri de Date\\
\textsc{Universiatea din Bucuresti}
}

\date{\today} %% Change "\today" by another date manually
%% -----------------------------
%% -----------------------------

%% %%%%%%%%%%%%%%%%%%%%%%%%%
\begin{document}
\setlength{\droptitle}{-5em}    
%% %%%%%%%%%%%%%%%%%%%%%%%%%
\maketitle

% --------------------------
% Start here
% --------------------------

% %%%%%%%%%%%%%%%%%%%
\section*{Problema 1}
% %%%%%%%%%%%%%%%%%%%
{\bfseries Demonstrati ca orice algoritm care construieste un arbore binar de  cautare cu $n$ numere  ruleaza in timp $\Omega(n \log{n})$.}

Luam in considerare doar cazul cel mai nefavorabil al algoritmilor pentru complexitatile descrise mai jos. \\
Consideram urmatoarele propozitii ca fiind adevarate:
\begin{enumerate}
    \item Parcurgerea unui arbore binar de cautare pentru a extrage elementele sortate ruleaza in $\Theta(n)$.
    \item Niciun algoritm de sortare nu poate rula in mai putin de $\Omega(n \log{n})$.
\end{enumerate}
Presupunem prin absurd ca exista un algoritm de constructie care ruleaza in mai putin de $\Omega(n \log{n})$. Atunci, conform propozitiei 1 rezulta ca putem sorta orice sir de numere in mai putin de  $\Omega(n \log{n})$, deci propozitia 2 este falsa. $\bot$ \\
Din contradictia de mai sus, deducem ca orice algoritm de constructie al unui arbore binar de cautare ruleaza in $\Omega({n \log{n}})$.

\section*{Problema 2}
{\bfseries Demonstrati ca daca $f(n) = \Theta(g(n))$ si $g(n) = \Theta(h(n))$ atunci $f(n) = \Theta(h(n))$}.

Conform definitiei, avem: \\
\begin{equation*}
\begin{split}
f(n) = \Theta(g(n)) \iff \forall c_{g1},c_{g2}\; \exists n_{g0} \in \mathbb{N} \; st. \; \forall n > n_{g0} \implies 0 \leq c_{g1} \cdot g(n) \leq f(n) \leq c_{g2} \cdot g(n) \\
g(n) = \Theta(h(n)) \iff \forall c_{h1},c_{h2}\; \exists n_{h0} \in \mathbb{N} \; st. \; \forall n > n_{h0} \implies 0 \leq c_{h1} \cdot h(n) \leq g(n) \leq c_{h2} \cdot h(n)
\end{split}
\\
\end{equation*}
Substituind, rezulta: \\
\begin{multline*}
0 \leq c_{h1} c_{g1} \cdot h(n) \leq c_{g1} \cdot g(n) \leq f(n) \leq c_{g2} \cdot g(n) \leq c_{h2} c_{g2} \cdot h(n) \implies \\
\implies 0 \leq c_{h1} c_{g1} \cdot h(n) \leq f(n)  \leq c_{h2} c_{g2} \cdot h(n) \\
\therefore \forall c_{1} = c_{g1} c_{h1},c_{2} = c_{g2} c_{h2} \; \exists n_{0} = max(n_{g0}, n_{h0}) \in \mathbb{N} \; st. \; \forall n > n_{0} \implies \\ \implies 0 \leq c_{1} \cdot h(n) \leq f(n)  \leq c_{2} \cdot h(n) \Longleftrightarrow f(n) = \Theta(h(n)) \\
\qedsymbol
\end{multline*}

\newpage 

\section*{Problema 3}
{\bfseries Demonstrati ca $\log{n} = o(\sqrt{n})$.} \\
Functiile $\log{n}$ si $\sqrt{n}$ sunt elementare, deci putem garanta existenta limitei:
\begin{equation*}
\begin{split}
    L = \lim_{n \to +\infty} {\frac{\log{n}}{\sqrt{n}}} = \frac{+\infty}{+\infty} \xRightarrow{L'H\hat{o}pital} L = \lim_{n \to +\infty} {\frac{\frac{1}{n \ln{2}}}{\frac{1}{2 \sqrt{n}}}} = \lim_{n \to +\infty} \frac{2\sqrt{n}}{n\ln{2}} = 0 \implies \\
    \implies \forall \epsilon > 0 \; \exists n_0 \in \mathbb{N} \; st. \; \forall n > n_0 \implies \frac{\log{n}}{\sqrt{n}} - 0 < \epsilon \implies 0 \leq \log{n} < \epsilon \cdot \sqrt{n} \\
\end{split}
\end{equation*}
\begin{equation*}
    \pushQED{\qed} 
    \therefore \log{n} = o(\sqrt{n}) \qedhere
    \popQED
\end{equation*}

\section*{Problema 4}
{\bfseries Se da un sir cu $n$ numere de la 1 la $n$, cu exceptia unui singur numar care apare de doua ori. Determinati numarul care apare de doua ori.} \\

Fie $x_n$ sirul dat. Notam $S_{x_k} = \sum\limits_{k=1}^{n} {x_k} = 1 + 2 + ... + 2 \cdot i + ... + n - 1$, unde $i$ este numarul care apare de doua ori si $S_{n} = \sum\limits_{k=1}^{n} {k} = \frac{n(n+1)}{2}$. \\
Atunci, $S_{x_k} - S_{n - 1} = i$. \\
Implementarea algoritmului, evident cu complexitatea in timp $\Theta(n)$: \\
\begin{verbatim}
FIND-DUPLICATE(X: the array, n: its length):
    sum_X = 0
    for i = 0 to n - 1
        sum_X += X[i]
    duplicate = sum_X - (n - 1) * n / 2
    return duplicate
\end{verbatim}

\section*{Problema 5}
{\bfseries{Fie X[1 :: n] si Y[1 :: n] doi vectori, fiecare continand $n$ numere \textit{sortate}. Prezentati un algoritm care sa gaseasca mediana celor $2n$ elemente. Mediana unei multimi de $n$ elemente este elementul de pe pozitia $[n/2]$ in sirul sortat.}} \\



\newpage

\section*{Problema 6}
{\bfseries{Se presupunem urmatoarele. Ati castigat la loterie si v-ati cumparat o vila pe care doriti sa o mobilati. Deoarece Ferrari-ul dvs. are capacitate limitata, doriti sa faceti cat mai putine drumuri de la magazin la vila. Mai exact, Ferrari-ul are capacitate $n$, iar dumneavoastra aveti de cumparat $k$ bunuri de dimensiune $x_1, x_2, ..., x_k$.} \\ 
Fie urmatorul algoritm greedy. Parcurgem bunurile in ordinea 1,2,...$k$ si incercam sa le punem in masina. In momentul in care un bun nu mai incape in masina, efectuam un transport si continuam algoritmul.}

\subsection*{Cerinta 1}
{\bfseries{Demonstrati ca algoritmul prezentat mai sus nu este optim.}} \\
Presupunem prin absurd ca algoritmul este optim si luam urmatorul sir ordonat de bunuri: $(3, 3, 2, 2)$ si $n=5$. Aplicand algoritmul obtinem 3 drumuri: $(3), (3), (2, 2)$. Rearanjand, putem obtine doar 2 drumuri: $(2, 3), (2, 3)$. \\ 
In concluzie, algoritmul \textbf{NU} este optim.

\subsection*{Cerinta 2}
{\bfseries{Fie OPT, numarul de drumuri in solutia optima. Demonstrati ca algoritmul greedy prezentat mai sus efectueaza cel mult 2OPT drumuri.}} \\

Consideram ca greutatea oricarui bun respecta conditia: $x_i \leq n$. \\
Vom nota greutatea totala a bunurilor din sirul $x_k$ cu $S_{x_k} = \sum\limits_{i = 1}^{k} {x_i}$. \\ 
Dat fiind un subsir $y_k$ a lui $x_k$ de bunuri transportate in $i$ zile, definim \textbf{capacitatea totala utilizata in $i$ zile:} $T_i = n \cdot i$, unde $T_i \geq S_{y_k}$ si \textbf{capacitatea totala irosita pentru a transporta $y_k$ : } $L_{i,y_k} = T_i - S_{y_k}$.
\begin{equation*}
    L_{i,x_k} = T_i - S_{x_k} \Longleftrightarrow T_i = L_{i,x_k} + S{x_k} \Longleftrightarrow n \cdot i = L_{i,x_k} + S{x_k} \Longleftrightarrow i = \frac{L_{i,x_k} + S{x_k}}{n}
\end{equation*}
Evident in cazul optim $i = OPT$. Cum $L_{i,x_k} \geq 0$, avem:
\begin{equation}
    OPT \geq \frac{S_{x_k}}{n}
\end{equation}
Luand un subsir oarecare $y_k$, daca subsirul $y_{k-1}$ poate fi transportat intr-un singur drum, dar $y_{k}$ nu, atunci $L_{1,y_{k-1}} < y_k$. Considerand toate cele $i$ subsiruri ale lui $x_k$, concluzionam ca $L_{i,x_k} < S_{x_k}$.
\begin{equation}
    L_{i,x_k} < S_{x_k} \Longleftrightarrow L_{i,x_k} + S_{x_k} < 2 \cdot S_{x_k} \Longleftrightarrow \frac{L_{i,x_k} + S_{x_k}}{n} < \frac{2 \cdot S_{x_k}}{n} \Longleftrightarrow i < \frac{2 \cdot S_{x_k}}{n}
\end{equation}
(1), (2) $\implies i < 2 \cdot OPT$, oricat de ineficient ar fi algoritmul folosit. \qed

\end{document}